\documentclass[12pt,a4paper,helvetica,titlepage]{article}
\usepackage[utf8]{inputenc}
\usepackage[spanish]{babel}
\usepackage[T1]{fontenc}
\usepackage[pdftex]{color,graphicx}
\usepackage[colorlinks=true,urlcolor=black]{hyperref}
\usepackage[hyphenbreaks,preserveurlmacro]{breakurl}
\usepackage{graphicx}
\usepackage{color}
\usepackage{xcolor}
\usepackage{listings}
 \lstset{
         basicstyle=\footnotesize\ttfamily, % Standardschrift
         numberstyle=\tiny,          % Stil der Zeilennummern
         numbersep=5pt,              % Abstand der Nummern zum Text
         tabsize=2,                  % Groesse von Tabs
         extendedchars=true,         %
         breaklines=true,            % Zeilen werden Umgebrochen
         keywordstyle=\color{red},
    	 frame=b,         
         keywordstyle=[1]\textbf,    % Stil der Keywords
 %        keywordstyle=[2]\textbf,    %
 %        keywordstyle=[3]\textbf,    %
 %        keywordstyle=[4]\textbf,   \sqrt{\sqrt{}} %
         stringstyle=\emph\ttfamily, % Farbe der String
         showspaces=false,           % Leerzeichen anzeigen ?
         showtabs=false,             % Tabs anzeigen ?
         xleftmargin=17pt,
         framexleftmargin=17pt,
         framexrightmargin=5pt,
         framexbottommargin=4pt,
         showstringspaces=false      % Leerzeichen in Strings anzeigen ?        
}


\usepackage{caption}
\DeclareCaptionFont{white}{\color{white}}
\DeclareCaptionFormat{listing}{\colorbox{gray}{\parbox{\textwidth}{#1#2#3}}}
\captionsetup[lstlisting]{format=listing,labelfont=white,textfont=white}



\renewcommand{\tablename}{Tabla}

\title{\Huge{Práctica 3: Threads en espacio de usuario.} \\ \normalsize{Diseño de Sistemas Operativos}}

\author{Pablo Castro Valiño~(pablo.castro1@udc.es)}


\date{Curso 2012-2013}
	
	
\begin{document}
	
	
\maketitle

\newpage

\tableofcontents

\newpage

\section{Introducción}
La utilización de threads a día de hoy es una práctica tremendamente común y fundamental
para el desarrollo de la informática actual. La necesidad de aumentar la paralelización
en los procesos ha llevado a la industria a hacer especial hincapié en el desarrollo de
este campo de la computación.

Nuestra práctica en concreto consiste en una implementación rudimentaria de threads en
espacio de usuario, los cuales utilizaremos en la parte básica para realizar el cálculo
simultáneo del factorial de tres números diferentes y en la parte optativa, para la
realización de un programa que realice doble buffering, en una primera parte de manera
síncrona y en una segunda de manera asíncrona.

\section{Procedimiento para la realización de la práctica.}

\subsection{Realización de un factorial en 3 threads.}
Para la realización de esta parte de práctica se ha creado el fichero
\textbf{factorial.c} (\ref{pre3}) con el total contenido de la misma.
En él encontramos diferenciadas las funciones que implementan los diferentes threads
(\emph{t0, t1, t2}), la función \emph{yield} que realiza el intercambio entre ellos y la
función \emph{adjust\_frame} que funciona como función auxiliar para recolocar el
\emph{stack pointer} cuando se realiza un salto.

A continuación vamos a comentar el funcionamiento de las diferentes funciones:

\begin{description}

    \item \textbf{t0, t1, t2} 
    El comportamiento de estas funciones es tremendamente simple, se limitan a realizar
el cálculo del factorial del número que reciben, de forma recursiva y haciendo continuas
llamadas a \emph{adjust\_frame} para cambiar entre los hilos.

    \item \textbf{adjust\_frame}
    Es la función que únicamente llama a \emph{Yield} y realiza un return.
    
    Se utiliza únicamente para conseguir que cuando se realice un salto y se modifique el
puntero base de la pila, la cima de la pila quede en el lugar donde queremos.

En circunstancias normales, el puntero de la cima de la pila va a quedarse muy arriba, de
forma que si introducimos esta función antes de volver a nuestro código, al realizar
\emph{return} de ella, se colocará el puntero cima de pila en donde está el base actual y
el base se colocará en la dirección adecuada del frame anterior, y de esta forma se
continuará ejecutando correctamente el código de la función.

    \item \textbf{Yield}
    Es la función que realiza la inicialización de los threads y que cambia entre ellos.
    
    Lo primero que hace es guardar el frame anterior y la dirección de retorno y
comprobar si es la primera ejecución de cada uno de los threads, para en ese caso,
lanzarlos.

También se selecciona mediante un bucle el thread hacia el que vamos a cambiar y se
comprueba que dicho thread elegido no ha terminado, en caso contrario se selecciona el
siguiente.

En el caso de que todos hayan terminado, se coloca la variable \textbf{ct} a 3 para
indicar que se debe de terminar y volver al frame de la función \emph{adjust\_frame} que
se lanzó desde el \emph{Main}, y por lo tanto terminar la ejecución del programa.

\end{description}

\subsection{Doble buffering}
Para la realización de esta parte de práctica se ha creado el fichero
\textbf{copia\_sincro.c} (\ref{pre4}) con el total contenido de la misma.\\

En él encontramos diferenciadas las funciones que implementan los diferentes threads
(\emph{t0, t1}), la función \emph{yield} que realiza el intercambio entre ellos y la
función \emph{slow} que funciona como función auxiliar realizando cálculos inservibles
para realentizar la ejecución.\\

El funcionamiento del programa es muy simple, se abre el fichero de tres modos distintos,
el primero para crear el archivo en el que se realizará la lectura y escritura en caso de
que no exista, la segunda en modo lectura y la tercera en modo escritura.\\

El thread \emph{t0} realiza la lectura del fichero en un buffer común mientras que
\emph{t1} se encarga de la escritura de dicho buffer en el fichero, de forma que cuando
terminemos se habrá leido del fichero el valor introducido, 5 veces y se habrá vuelto a
escribir en el fichero.
De esta forma el fichero \textbf{entrada.txt} contrendrá 'VALOR' 6 veces, el que hemos
introducido desde el \emph{Main} para inicializar el fichero y los 5 siguientes que hemos
leido y reescrito.

\newpage

\section{Índice de archivos adjuntos.}

\begin{enumerate}

    \item \textbf{Makefile} \label{pre0}
Se ha introducido un fichero \emph{Makefile} para agilizar la compilación. 

      \begin{itemize}

	  \item \textbf{make all}
	  Compila todos los ficheros.

          \item \textbf{make fact}
          Compila los ficheros que realizan el factorial.

          \item \textbf{make sinc}
	  Compila el fichero que realiza el doble buffering.

          \item \textbf{make clean}
	  Elimina todos los ficheros ejecutables.

      \end{itemize}

  \item \textbf{tests/programa1.c} \label{pre1}
  \\
Programa básico realizado en clase para testear el limite del stack.

  \item \textbf{tests/programa2.c} \label{pre2}
  \\
Programa básico realizado en clase para la realización de los threads, que únicamente 
escribe la letra 'a' o 'b' en función del thread en el que se encuentre, y lo hace de 
manera alternativa.

  \item \textbf{factorial.c} \label{pre3} 
  \\
Programa que realiza el factorial de forma recursiva en tres threads distintos. 
Para ello se sube por el stack haciendo llamadas recursivas en cada thread, de forma que 
el ultimo frame de cada uno, el valor del número del que se quiere hacer el factorial va a 
ser 0 y luego vuelve a bajar intercalando unos threads con otros para realizar el cálculo.

    \item \textbf{copia\_sincro.c} \label{pre4} 
	\\
Programa que crea dos threads, uno que escribe en un fichero y el otro que lee.

  \end{enumerate}

\end{document}